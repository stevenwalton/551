\documentclass[12pt,letter]{article}
\usepackage{geometry}\geometry{top=0.75in}
\usepackage{amsmath}
\usepackage{amssymb}
\usepackage{mathtools}
\usepackage{xcolor} % Color words
\usepackage{cancel} % Crossing parts of equations out
\usepackage{tikz}       % Drawing 
\usepackage{pgfplots}   % Other plotting
\usepgfplotslibrary{colormaps,fillbetween}
\usepackage{placeins}   % Float barrier
\usepackage{hyperref}   % Links
\usepackage{tikz-qtree} % Trees
\usepackage{graphicx}
\usepackage{subcaption}
\usepackage{multicol}
\usepackage{graphicx}   % For graphics
\usepackage{parcolumns}
\usepackage{listings}   % lstlisting
\usepackage{pdfpages}

\begin{document}
\title{CIS 551: Databases Final Project\\
\large Database of Climbing Routes}
\author{Steven Walton}
\maketitle

\section*{}
\begin{align*}
    &\text{Port number: }3875 \\
    &\text{Username: guest. No password}\\
    &\text{Database Name: climbing}\\
    &\text{URL:
    \href{https://ix.cs.uoregon.edu/~swalton2/551/Final/}{https://ix.cs.uoregon.edu/~swalton2/551/Final/}}\\
\end{align*}
\newpage
\section{Table of Contents}
\ref{sec:sum} Summary page      \hspace*{\fill}Page:\pageref{sec:sum}\\
\ref{sec:ld} Logical Design     \hspace*{\fill}Page:\pageref{sec:ld}\\
\ref{sec:apps} Applications     \hspace*{\fill}Page:\pageref{sec:apps}\\
\ref{sec:ug} User Guide         \hspace*{\fill}Page:\pageref{sec:ug}\\
\ref{sec:cot} Content of Tables \hspace*{\fill}Page:\pageref{sec:cot}\\
\ref{sec:imp} Implementations   \hspace*{\fill}Page:\pageref{sec:imp}\\
\ref{sec:con} Conclusion        \hspace*{\fill}Page:\pageref{sec:con}

\section{Summary}\label{sec:sum}

In this project I created a website similar to that of Mountain Project (links
on site). The purpose of this site is so that rock climbers can create a list of
routes. These routes will have the information necessary for them to know if
they can climb them and what to prepare for. Being community driven, users are
encouraged to add to routes and vote on the difficulty and likability of routes. 
They are also able to find pictures that have been taken. 

There are many applications that are within the program. The main applications
for the users are that they can search routes, see the entire listings, and
investigate individual routes. Users are also able to submit new routes. They
can even submit routes in countries, states, sites, etc that do not yet exist in
the database. If they call the submit route function then it will generate those
as well. Additionally, if a user specifies that a route has more than one pitch,
then the route page will automatically take them through the process of
specifying each pitch, then will bring them to the landing page as they submit
the last pitch. 

When searching a route or listing all of them, the route names are clickable
links. This will take the user to the respective route page and display the
relevant information for them, including any pictures. Additionally on this page
users are able to vote on the likability and difficulty of a particular route.

Users are also able to generate their own profiles. Their usernames are also
clickable links within the user listing directory. Their page shows their name,
a description, and what photos they have submitted to the site. 

\section{Logical Design}\label{sec:ld}
\includegraphics[width=\textwidth]{ER.png}

\section{List of Applications}\label{sec:apps}

\begin{itemize}
    \item Submissions
        \begin{itemize}
            \item Country
            \item State
            \item Site
            \item Area
            \item Route
            \item Pitch
            \item User
            \item Picture
        \end{itemize}
    \item Update Information
        \begin{itemize}
            \item Route Difficulty
            \item Route Likability
        \end{itemize}
    \item User Pages
    \item Route Pages
    \item Search
        \begin{itemize}
            \item By Country
            \item By State
            \item By Site
            \item By Area
            \item By Route
            \item Or by combination
        \end{itemize}
    \item User Directory
    \item Route Directory
    \item Random Picture Display on Landing Page
    \item API
\end{itemize}

\section{User's Guide}\label{sec:ug}

This website was designed to make things easy on the user, and thus is expected
to not need much guidance. 

On the landing page the user is presented with the ability to search, add a
user, see the user directory, see the route directory (sorted by popularity and
then difficulty), and able to add new routes. The landing page also shows a
random picture from the database.

In searching and the route directory listing users are presented with the
country, state, site, area, route, type, number of pitches, difficulty, and
likability of routes. The route names are buttons which will bring them to the
specific route page. 

On the route page they are presented with the same information, but additionally
with a description of the route and the approach. Users are also able to vote on
the difficulty and likability of the route. These values default to the already
established values. Both are averages based on the number of times that users
have voted on a route. Pictures from the route are also displayed. If a route
has multiple pitches, then these pitches are also displayed to the user.

A user is also able to add themselves with a short description. In the user
directory we can see the name and description of the user. The name is a button
which will generate a page that shows the same information and the images that
the user has submitted.

Note: there is a backup of the database. If anything goes wrong, please resource
this one.

\subsection{API}
In the \textit{/sec} folder we find all the class based object and we can thus
use these API calls for testing and breaking things. In this section we
highlight some of these functions.

\subsubsection{connect.php}
This is root function and all other functions will call from it. This just
handles the connection information and only has one function ``connect".

\subsubsection{countryFunctions.php, stateFunctions.php, siteFunctions.php,
areaFunctions.php}
This file has functions related to the country table. Many of these are useful
for testing and others are used within the website. One can either just print
out all countries with ``listCountries()'' or receive back an array with them
with ``getAllCountries()". Functions also exist to get the name or ID of the
country, add a new one, or delete by name or ID. The same functions exist for
states, sites, and areas. 

It should also be noted that there is a recursive process here. There is a
hierarchy within these files: country > state > site > area. If one creates an
area then a site, state, and country will also be created (country, then state,
then site, then finally area). Dynamically creating objects like this ensures
that we have a consistent and logical database. All add statements are done with
PDO::prepare. The execute statements are then in try-catch blocks to preserve
consistency.

\subsubsection{routeFunctions.php}
The route function is the top of the hierarchy in the above chain. Creating a
route will force the creation of all other elements if they are not already
created. It also has similar functions to the above table APIs. 

Route also has some other unique functions to it. These include searching
through routes with any combination of: ``country", ``state", ``site", ``area",
``likability", ``difficulty", ``type", or ``number of pitches". Additionally,
one can just call functions such as ``getRoutesIn\{Country,State,Area,etc.\}.
The Route API also allows the user to update the likability with
``updateLikability(int vote, int idRoute)" and difficulty with
``updateDifficulty(float vote, int idRoute)". These functions also count each
time they were called, and average the inputs given. 

\subsubsection{pitchFunctions.php}
This API only allows for a user to access the pitches in a route or to add a new
pitch, supplying the route ID. 

\subsubsection{pictureFunctions.php}
This API has similar functions as the others. What is unique in here is that we
can grab a random picture either from the entire set of pictures or from a
route. We can also easily obtain pictures that are uploaded by a single user.
Users can also upload pictures that are not connected to routes (this won't show
on the landing page).

This API also handles the bridge table operations. (See ER diagram)

\subsubsection{userFunctions.php}
This has similar calls to the rest of the API functions. 

\subsubsection{basicFunctions.php}
The Basic API handles 2 things. The first is to print all the tables. This is
for testing and investigation purposes only. The other part is that this handles
the conversion back and forth between the floating point value for difficulty
that is stored within the route table and the Yosemite Decimal System. The use
of these functions allows us to talk in a language that climbers understand.

\section{Contents of Tables}\label{sec:cot}
\FloatBarrier

\begin{figure}[ht]
\includegraphics[width=\textwidth]{../media/country.png}
\end{figure}
%
\begin{figure}[ht]
\includegraphics[width=\textwidth]{../media/state.png}
\end{figure}
%
\begin{figure}[ht]
\includegraphics[width=\textwidth]{../media/site.png}
\end{figure}
%
\begin{figure}[ht]
\includegraphics[width=\textwidth]{../media/area.png}
\end{figure}
%
\begin{figure}[ht]
\includegraphics[width=\textwidth]{../media/route.png}
\end{figure}
%
\begin{figure}[ht]
\includegraphics[width=\textwidth]{../media/routeHasPics.png}
\end{figure}
%
\begin{figure}[ht]
\includegraphics[width=\textwidth]{../media/routeHasPitch.png}
\end{figure}
%
\begin{figure}[ht]
\includegraphics[width=\textwidth]{../media/pitch.png}
\end{figure}
%
\begin{figure}[ht]
\includegraphics[width=\textwidth]{../media/users.png}
\end{figure}
%
\begin{figure}[ht]
\includegraphics[width=\textwidth]{../media/pictures.png}
\end{figure}
\FloatBarrier

\section{Implementation of Code}\label{sec:imp}
All code is available on GitHub at
\href{https://github.com/stevenwalton/551/tree/master/Final}{https://github.com/stevenwalton/551/tree/master/Final}

Functions for relevant tables are located in the \textit{/src} directory. All
API calls will be located within these files. Other
functions are located in the \textit{/scripts} directory, as most of these are
single time use pages and display different things based on user input. Some of
them are just helper pages and will redirect back to the relevant location.

The database and MySQL Workbench files are located in \textit{/database}.

\section{Conclusion}\label{sec:con}
I believe that this is a workable database and would represent a state of the
art system! If it were 1995. Overall, I believe that there is a lot of utility
in the user generated system and having users be able to contribute to a
database freely. This site accomplishes these goals.

If I had more time, I would have locked some things down in the database. It is
completely unprotected at this point. This means that someone can come in and
drop all the tables. Luckily everything is not important and everything is
backed up (both the tables and the code). Given more time I could have created
API calls and a cron job to automatically backup in a given time window. This
would be done if it was not just a school project. 

One that that I really wanted to implement was to be able to navigate through
countries, states, and so on. Seeing the routes along the way. I had started to
implement this and you can see some of it in index\_test.php, but IX was giving
me too many problems. I lost a lot of hours trying to implement this so I
carried on with the rest of the design and left this out.

\end{document}
