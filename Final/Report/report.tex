\documentclass[12pt,letter]{article}
\usepackage{geometry}\geometry{top=0.75in}
\usepackage{amsmath}
\usepackage{amssymb}
\usepackage{mathtools}
\usepackage{xcolor} % Color words
\usepackage{cancel} % Crossing parts of equations out
\usepackage{tikz}       % Drawing 
\usepackage{pgfplots}   % Other plotting
\usepgfplotslibrary{colormaps,fillbetween}
\usepackage{placeins}   % Float barrier
\usepackage{hyperref}   % Links
\usepackage{tikz-qtree} % Trees
\usepackage{graphicx}
\usepackage{subcaption}
\usepackage{multicol}
\usepackage{graphicx}   % For graphics
\usepackage{parcolumns}
\usepackage{listings}   % lstlisting
\usepackage{pdfpages}

\begin{document}
\title{CIS 551: Databases Final Project\\
\large Database of Climbing Routes}
\author{Steven Walton}
\maketitle

\section*{}
\begin{align*}
    &\text{Port number: }3875 \\
    &\text{Username: guest. No password}\\
    &\text{Database Name: climbing}\\
    &\text{URL:
    \href{https://ix.cs.uoregon.edu/~swalton2/551/Final/}{https://ix.cs.uoregon.edu/~swalton2/551/Final/}}\\
\end{align*}
\newpage
\section{Table of Contents}

\section{Summary}
In this project I created a website similar to that of Mountain Project (links
on site). The purpose of this site is so that rock climbers can create a list of
routes. These routes will have the information necessary for them to know if
they can climb them and what to prepare for. Being community driven, users are
encouraged to add to routes and vote on the difficulty and likability of routes. 
They are also able to find pictures that have been taken. 

There are many applications that are within the program. The main applications
for the users are that they can search routes, see the entire listings, and
investigate individual routes. Users are also able to submit new routes. They
can even submit routes in countries, states, sites, etc that do not yet exist in
the database. If they call the submit route function then it will generate those
as well. Additionally, if a user specifies that a route has more than one pitch,
then the route page will automatically take them through the process of
specifying each pitch, then will bring them to the landing page as they submit
the last pitch. 

When searching a route or listing all of them, the route names are clickable
links. This will take the user to the respective route page and display the
relevant information for them, including any pictures. Additionally on this page
users are able to vote on the likability and difficulty of a particular route.

Users are also able to generate their own profiles. Their usernames are also
clickable links within the user listing directory. Their page shows their name,
a description, and what photos they have submitted to the site. 

\section{Logical Design}
\includegraphics[width=\textwidth]{ER.png}

\section{List of Applications}

\begin{itemize}
    \item Submissions
        \begin{itemize}
            \item Country
            \item State
            \item Site
            \item Area
            \item Route
            \item Pitch
            \item User
            \item Picture
        \end{itemize}
    \item Update Information
        \begin{itemize}
            \item Route Difficulty
            \item Route Likability
        \end{itemize}
    \item User Pages
    \item Route Pages
    \item Search
        \begin{itemize}
            \item By Country
            \item By State
            \item By Site
            \item By Area
            \item By Route
            \item Or by combination
        \end{itemize}
    \item User Directory
    \item Route Directory
    \item Random Picture Display on Landing Page
\end{itemize}

\section{User's Guide}
This website was designed to make things easy on the user, and thus is expected
to not need much guidance. 

On the landing page the user is presented with the ability to search, add a
user, see the user directory, see the route directory (sorted by popularity and
then difficulty), and able to add new routes. The landing page also shows a
random picture from the database.

In searching and the route directory listing users are presented with the
country, state, site, area, route, type, number of pitches, difficulty, and
likability of routes. The route names are buttons which will bring them to the
specific route page. 

On the route page they are presented with the same information, but additionally
with a description of the route and the approach. Users are also able to vote on
the difficulty and likability of the route. These values default to the already
established values. Both are averages based on the number of times that users
have voted on a route. Pictures from the route are also displayed. If a route
has multiple pitches, then these pitches are also displayed to the user.

A user is also able to add themselves with a short description. In the user
directory we can see the name and description of the user. The name is a button
which will generate a page that shows the same information and the images that
the user has submitted.

\section{Contents of Tables}

\section{Implementation of Code}
All code is available on GitHub at
\href{https://github.com/stevenwalton/551/tree/master/Final}{https://github.com/stevenwalton/551/tree/master/Final}

Functions for relevant tables are located in the \textit{/src} directory. Other
functions are located in the \textit{/scripts} directory, as most of these are
single time use pages and display different things based on user input. Some of
them are just helper pages and will redirect back to the relevant location.

The database and MySQL Workbench files are located in \textit{/database}.

\section{Conclusion}

\end{document}
